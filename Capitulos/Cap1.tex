\chapter{Introducci\'on}
\section{Introducci\'on}
En el panorama de salud actual, las enfermedades cr\'onicas han tomado un papel preponderante al ser padecimientos que no tienen cura, que se prolongan de manera indefinida y que requieren de atenci\'on y monitoreo constante.  Uno de los principales problemas de estos padecimientos consiste en la gran carga econ\'omica que \'estos presentan, junto con sus complicaciones; tanto a pacientes como a los proveedores de servicios de salud. Es por esto que calcular duraciones y costos futuros se vuelve de particular importancia.\\
\\
Por lo anterior, el objetivo de este trabajo de investigaci\'on es desarrollar una metodolog\'ia para poder predecir duraciones y costos futuros con la particularidad de no suponer independencia entre las ocurrencias individuales. Con la ayuda del modelo propuesto ser\'a posible caracterizar las trayectorias individuales de pacientes, para as\'i poder predecir con precisi\'on los costos y duraciones futuras de manera particular, haciendo posible estimaciones agregadas.\\
\\
En el desarrollo de este modelo, se supone que las ocurrencias individuales en el tiempo se conforman de dos variables: la duraci\'on de un individuo en un diagn\'ostico y el costo del tratamiento para dicho diagn\'ostico. Estas variables se asumen dependientes entre ellas, el costo depende de la duraci\'on , y se asumen dependientes en el tiempo. Tomando las caracter\'isticas del modelo propuesto, diversas instituciones, tanto p\'ublicas como privadas; podr\'an calcular reservas o primas de seguro de manera m\'as precisa para el caso de padecimientos cr\'onicos degenerativos.\\
\\
\section{Escasez de datos}
Este trabajo de investigaci\'on se concentra principalmente en pacientes con diabetes mellitus tipo-II, debido a que es el padecimiento cr\'onico que m\'as ha incrementado su prevalencia en los \'ultimos a$\tilde{n}$os. Aunque se cuenta con mucha informaci\'on sobre la prevalencia y los costos de la diabetes mellitus tipo-II en el mundo, la mayor\'ia de esta informaci\'on es obtenida de distintas fuentes aunque se refieran al mismo pa\'is, en el caso particular de M\'exico, se han realizado varios estudios tratando de estimar la prevalencia y los costos de este padecimiento; varios de ellos analizados en la secci\'on anterior.\\
\\
De acuerdo con \cite{barquera2013diabetes}, los mayores registros sobre las trayectorias de las enfermedades y los gastos asociados a los tratamientos son elaborados por las distintas instituciones p\'ublicas de salud como el IMSS, ISSSTE y la Secretar\'ia de Salud, que son exclusivas para uso interno administrativo por lo que no se ha podido elaborar una base de datos consistente y precisa con la que se pueda realizar un an\'alisis epidemiol\'ogico y estad\'istico. Aunque se cuenta con la ENSANUT como un estudio general para la prevalencia y costos y con el INEGI para las estad\'isticas de mortalidad desagregadas por causa, \'estas no cuentan con la informaci\'on individual de los pacientes y sus trayectorias. Es por esta raz\'on que este trabajo de investigaci\'on no se pudo realizar con informaci\'on sobre pacientes de M\'exico.\\
\\
El problema de la falta de bases de datos con las trayectorias de los pacientes de cualquier padecimiento no solo afecta a M\'exico. Tambi\'en a pa\'ises como Estados Unidos de Norteam\'erica donde existen tantas instancias que estudian el padecimiento de diabetes como la AHRQ, la Asociaci\'on Americana de la Diabetes, el Centro para el Control y Prevenci\'on de Enfermedades; que resulta inexistente una base consolidada de datos con la informaci\'on de los diagn\'osticos y tratamientos de los pacientes. Incluso, la OMS solo cuenta con la informaci\'on general de prevalencia, costos y mortalidad con ayuda de la Federaci\'on Internacional de la Diabetes.\\
\\
Sin embargo, el NHS de Reino Unido ha hecho un esfuerzo por elaborar este tipo de bases de datos con las trayectorias completas de los pacientes. Es por este motivo que en este trabajo de investigaci\'on se utilizar\'a esta base de datos para la implementaci\'on del modelo. Esta base de datos comprende las trayectorias de padecimiento a trav\'es del tiempo de 500 pacientes.\\
\\
En el siguiente cap\'itulo de este trabajo de investigaci\'on se elaborar\'a con detalle el modelo general de probabilidad que caracteriza las trayectorias individuales de los pacientes mediante un proceso de duraci\'on marcada integrando variables latentes. Una vez que el modelo se ha descrito, el cap\'itulo tres se concentra en sentar las bases anal\'iticas necesarias para poder hacer inferencia sobre observaciones futuras. En el cap\'itulo cuatro, el \'ultimo de desarrollo te\'orico, se especifica el m\'etodo de estimaci\'on bayesiano que se utilizar\'a en este trabajo de investigaci\'on y el algoritmo de inferencia basado en esta metodolog\'ia. Por \'ultimo, el cap\'itulo cinco explora una aplicaci\'on pr\'actica del problema con la base de datos provenientes del Servicio Nacional de Salud del Reino Unido y las correspondientes conclusiones del trabajo. 