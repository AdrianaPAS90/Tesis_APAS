\chapter{El Algoritmo EM}
El algoritmo EM es un algoritmo para calcular el estimador de m\'axima verosimilitud, que de acuerdo con \cite{held2014applied}, se define como
\begin{defi}
El Estimador de M\'axima Verosimilitud (EMV) $\hat{\theta}_{MV}$ del par\'ametro $\theta$ se obtiene maximizando la funci\'on de verosimilitud.
\begin{align*}
\hat{\theta}_{MV}=max_{\theta \in \Theta} L(\theta)
\end{align*} 
\end{defi}
Seg\'un \cite{dempster1977maximum} el algoritmo EM calcula el EML mediante iteraciones, cada iteraci\'on consiste en un paso d\'onde se calcula la esperanza y en otro  se maximiza la misma, de ah\'i el nombre de EM. Este algoritmo se relaciona con las variables latentes suponiendo dos variables $x$ y $y$ las cuales se relacionan $x \to y(x)$, donde $y$ son los datos observables.\\
\\
De este modo, an\'alogamente a lo expresado en el cap\'itulo anterior por \cite{pitt2002constructing}, se proponen las siguientes funciones de densidad $f(x|\phi)$ y $g(y|\phi)$; en las cuales, de acuerdo a \cite{dempster1977maximum} los datos completos (variables latentes) $f(x|\cdot)$ se relacionan con los datos incompletos (variables observadas) $g(y|\cdot)$ mediante
\begin{align*}
g(y|\phi)=\int_{\chi(y)} f(x|\phi)dx
\end{align*}
El algoritmo EM se dedica a encontrar un valor de $\phi$ que maximice $g(y|\phi)$ dada la $y$ observada usando la familia asociada de $f(x|\phi)$. Una de las caracterizaciones m\'as simples supone $\phi^{(p)}$ es el valor actual de $\phi$ despu\'es de $p$ iteraciones y $t(x)$ como el estad\'istico suficientes de los datos completos, es decir, el estimador de la variable latente; por lo que la siguiente iteraci\'on se puede desglosar en los siguientes dos pasos:
\begin{itemize}
\item Paso E: Estimar los estad\'isticos suficientes de los datos completos.
	\begin{align*}
	t^{(p)}=E[t(x)|y,\phi^{(p)}]
	\end{align*}
\item Paso M: Determinar $\phi^{(p+1)}$ como soluci\'on a la ecuaci\'on
	\begin{align*}
	E[t(x)|\phi]=t^{(p)}
	\end{align*}
\end{itemize}
Es decir, que si suponemos que $t^{(p)}$ es el estad\'istico suficiente calculado de $x$ observada en la distribuci\'on $f(x|\phi)$ entonces la ecuaci\'on definida en el Paso M se define como el EMV. Este concepto se hace general al definir la siguiente funci\'on
\begin{align*}
Q(\phi'|\phi)=E[log f(x|\phi')|y,\phi]
\end{align*}
Esta funci\'on se asume que existe para toda pareja $(\phi',\phi)$. Se define la iteraci\'on EM para $\phi^{(p)} \to \phi^{(p+1)}$,
\begin{itemize}
\item Calcular $Q(\phi,\phi^{(p)})$.
\item Determinar $\phi^{(p+1)}$ tal que maximice $Q(\phi,\phi^{(p)})$.
\end{itemize}
La idea central es tomar una $\phi'$ que maximice $log f(x|\phi)$, dado que esta distribuci\'on y su correspondiente logaritmo no necesariamente se conoce, se puede maximizar los datos observados y $\phi^{(p)}$.\\
\\
El algoritmo EM es muy \'util pues por su estructura iterativa puede dar resultados a modelos de probabilidad muy complejos, adem\'as de que al igual que el Muestreador de Gibbs, utiliza una estructura subyacente o de variables latentes.
