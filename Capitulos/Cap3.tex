\chapter{Inferencia y Predicci\'on}
\section{Introducci\'on}
En el cap\'itulo anterior se describi\'o el modelo general de probabilidad que describe el proceso de duraciones y costos en un padecimiento cr\'onico degenerativo, por lo que el siguiente paso ser\'ia hacer inferencia sobre el mismo. El objetivo de hacer inferencia es predecir futuras observaciones en base a los datos ya observados. En este cap\'itulo se sentar\'an las bases para realizar esta inferencia.\\
\\
El primer paso para realizar inferencia es la construcci\'on de la funci\'on de verosimilitud, en este caso extendida a las variables latentes y a los par\'ametros correspondientes a sus distribuciones. Una vez que se determinaron las funciones de verosimilitud, se analizan los m\'etodos de estimaci\'on que se podr\'ian usar para la predicci\'on de futuras observaciones. De este modo se puede completar la parte te\'orica de este trabajo de investigaci\'on.
\section{Verosimilitud Extendida}
Como especificado en la secci\'on anterior, una vez que el modelo de probabilidad describe de manera precisa los datos del problema podemos empezar a hacer inferencia sobre observaciones futuras. La base sobre la que se puede hacer inferencia en base a los datos ya observados es la funci\'on de verosimilitud de la funci\'on de probabilidad.\\
\\
La funci\'on de verosimilitud, seg\'un \cite{held2014applied}, se define como
\begin{defi}
La funci\'on de verosimilitud $V(\theta)$ es la funci\'on masa o la funci\'on de densidad de los datos observados $x$, entendidos en funci\'on del par\'ametro desconocido $\theta$.
\end{defi}
En este caso las variables observables se definen en funci\'on de las variables latentes, estas a su vez se describen en funci\'on de sus par\'ametros. De esto se desprende la noci\'on de verosimilitud extendida para incluir las variables latentes. Como se especifica en \cite{pitt2002constructing}, la construcci\'on de la funci\'on de verosimilitud resulta sencilla, incluso intuitiva. Sin embargo, la estimaci\'on de los par\'ametros mediante m\'axima verosimilitud no es tan sencilla pues no tiene una soluci\'on que se pueda expresar de manera anal\'itica cerrada. Usando esta construcci\'on, se escribe una funci\'on de verosimilitud para el modelo general de probabilidad de duraciones y costos\\
\begin{eqnarray}
V(\{\theta_j\},\{\gamma_j\},\{d_j\},\{c_j\}) &=& f(d_1|\theta_1)f(c_1|d_1,\gamma_1)f(\gamma_1) \times\nonumber\\
&& \times \prod_{j=2}^{N(t)} f(d_j|\theta_j)f(\theta_j|d_{j-1})f(c_j|d_j,\gamma_j)f(\gamma_j|c_{j-1})\nonumber
\end{eqnarray}
Para poder calcular la funci\'on de verosiimilitud que permite hacer inferencia, es necesario conocer las distribuciones de las variables latentes en base a las observaciones anteriores para ambas variables observables, duraciones y costos.\\
\\Para la primera variable observable se toma en cuenta la relaci\'on $f_{\theta|d}(\theta|d)\propto f_{d|\theta}(d|\theta)f(\theta)$ y que $d|\theta \sim Gamma(d|\alpha_d,\theta)$ y $\theta \sim Gamma(\theta|\alpha_\theta,\beta_\theta)$
\begin{eqnarray*}
f_{\theta|d}(\theta|d) &\propto& \frac{\theta^{\alpha_d}}{\Gamma(\alpha_d)} d^{\alpha_d-1} \exp{-\{\theta d\}} \times \frac{\beta_\theta^{\alpha_\theta}}{\Gamma(\alpha_\theta)} \theta^{\alpha_\theta-1} \exp{-\{-\beta_\theta \theta\}}\\
&=&\frac{\beta_\theta^{\alpha_\theta}}{\Gamma(\alpha_d)\Gamma(\alpha_\theta)} \quad d^{\alpha_d-1} \quad\theta^{\alpha_d+\alpha_\theta-1} \quad \exp{\{-\theta(d+\beta_\theta)\}}\\
&\propto& \theta^{\alpha_d+\alpha_\theta-1} \quad \exp{\{-\theta(d+\beta_\theta)\}}\\
&\Rightarrow& \theta|d \sim Gamma(\alpha_d+\alpha_\theta,d+\beta_\theta)
\end{eqnarray*}
\\
Para la variable de duraciones, la distribuci\'on de la variable latente que depende de la observaci\'on se puede expresar de una manera anal\'itica cerrada como la distribuci\'on Gamma. An\'alogamente, para la variable de costos se vuelve a tomar en cuenta la misma relaci\'on y las distribuciones $c|d,\gamma \sim Weibull(c|d,\gamma)$ y $\gamma \sim InvGamma(\gamma|\alpha_\gamma,\beta_\gamma)$
\begin{eqnarray*}
f_{\gamma|d,c}(\gamma|d,c) &\propto& \frac{d}{\gamma^d}\quad c^{d-1} \exp{\{-(\frac{c}{\gamma})^d\}} \times \frac{\beta_\gamma^{\alpha_\gamma}}{\Gamma(\alpha_\gamma)}\quad (\frac{1}{\gamma})^{\alpha_\gamma+1} \exp{\{-(\frac{\beta_\gamma}{\gamma})\}}\\
&=&\frac{d\beta_\gamma^{\alpha_\gamma} c^{d-1}}{\Gamma(\alpha_\gamma)}\quad (\frac{1}{\gamma})^{d+\alpha_\gamma+1}\exp{-((\frac{\beta_\gamma}{\gamma})+(\frac{c}{\gamma})^d)}\\
&\propto&(\frac{1}{\gamma})^{d+\alpha_\gamma+1}\exp{-((\frac{\beta_\gamma}{\gamma})+(\frac{c}{\gamma})^d)}
\end{eqnarray*}
\\
Para la variable de costos, la distribuci\'on de la variable latente seg\'un la observaci\'on anterior no tiene una forma anal\'itica cerrada como distribuci\'on, sin embargo, el kernel se puede simular con un slice sampler; este m\'etodo se explicar\'a con detalle en la siguiente secci\'on. \\
\\
Los par\'ametros a estimar son aquellos correspondientes a las variables latentes y a las variables observables del modelo. Dado el uso de las variables latentes no existir\'a un \'optimo global para las variables, raz\'on por la cual se requiere el uso de m\'etodos num\'ericos. \cite{pitt2002constructing} especifica que la estimaci\'on de m\'axima verosimilitud puede resolverse mediante el algoritmo EM, aunque tambi\'en, debido a que las densidades son dos condicionales de la densidad conjunta puede ligarse con el Muestreador Gibbs.\\
\\
El algoritmo EM es un algoritmo para calcular el estimador de m\'axima verosimilitud, que de acuerdo con \cite{held2014applied}, se define como
\begin{defi}
El Estimador de M\'axima Verosimilitud (EMV) $\hat{\theta}_{MV}$ del par\'ametro $\theta$ se obtiene maximizando la funci\'on de verosimilitud.
\begin{align*}
\hat{\theta}_{MV}=max_{\theta \in \Theta} L(\theta)
\end{align*} 
\end{defi}
Seg\'un \cite{dempster1977maximum} el algoritmo EM calcula el EML mediante iteraciones, cada iteraci\'on consiste en un paso d\'onde se calcula la esperanza y en otro  se maximiza la misma, de ah\'i el nombre de EM. Este algoritmo se relaciona con las variables latentes suponiendo dos variables $x$ y $y$ las cuales se relacionan $x \to y(x)$, donde $y$ son los datos observables.\\
\\
De este modo, an\'alogamente a lo expresado en el cap\'itulo anterior por \cite{pitt2002constructing}, se proponen las siguientes funciones de densidad $f(x|\phi)$ y $g(y|\phi)$; en las cuales, de acuerdo a \cite{dempster1977maximum} los datos completos (variables latentes) $f(x|\cdot)$ se relacionan con los datos incompletos (variables observadas) $g(y|\cdot)$ mediante
\begin{align*}
g(y|\phi)=\int_{\chi(y)} f(x|\phi)dx
\end{align*}
El algoritmo EM se dedica a encontrar un valor de $\phi$ que maximice $g(y|\phi)$ dada la $y$ observada usando la familia asociada de $f(x|\phi)$. Una de las caracterizaciones m\'as simples supone $\phi^{(p)}$ es el valor actual de $\phi$ despu\'es de $p$ iteraciones y $t(x)$ como el estad\'istico suficientes de los datos completos, es decir, el estimador de la variable latente; por lo que la siguiente iteraci\'on se puede desglosar en los siguientes dos pasos:
\begin{itemize}
\item Paso E: Estimar los estad\'isticos suficientes de los datos completos.
	\begin{align*}
	t^{(p)}=E[t(x)|y,\phi^{(p)}]
	\end{align*}
\item Paso M: Determinar $\phi^{(p+1)}$ como soluci\'on a la ecuaci\'on
	\begin{align*}
	E[t(x)|\phi]=t^{(p)}
	\end{align*}
\end{itemize}
Es decir, que si suponemos que $t^{(p)}$ es el estad\'istico suficiente calculado de $x$ observada en la distribuci\'on $f(x|\phi)$ entonces la ecuaci\'on definida en el Paso M se define como el EMV. Este concepto se hace general al definir la siguiente funci\'on
\begin{align*}
Q(\phi'|\phi)=E[log f(x|\phi')|y,\phi]
\end{align*}
Esta funci\'on se asume que existe para toda pareja $(\phi',\phi)$. Se define la iteraci\'on EM para $\phi^{(p)} \to \phi^{(p+1)}$,
\begin{itemize}
\item Calcular $Q(\phi,\phi^{(p)})$.
\item Determinar $\phi^{(p+1)}$ tal que maximice $Q(\phi,\phi^{(p)})$.
\end{itemize}
La idea central es tomar una $\phi'$ que maximice $log f(x|\phi)$, dado que esta distribuci\'on y su correspondiente logaritmo no necesariamente se conoce, se puede maximizar los datos observados y $\phi^{(p)}$.\\
\\
El algoritmo EM es muy \'util pues por su estructura iterativa puede dar resultados a modelos de probabilidad muy complejos, adem\'as de que al igual que el Muestreador de Gibbs, utiliza una estructura subyacente o de variables latentes. En la siguiente secci\'on se explorar\'a con detalle el Muestreador de Gibbs y sus implicaciones con el modelo general de probabilidad relativo a este trabajo.

\section{Slice Sampler}

\section{El Muestreador de Gibbs}
Como mencionado en la secci\'on anterior, la estimaci\'on de par\'ametros del modelo general de probabilidad que se utilizar\'a en este trabajo no se puede hacer a trav\'es de m\'etodos anal\'iticos tradicionales, por lo que se utilizar\'an m\'etodos num\'ericos. Entre estos se encuentran el algoritmo EM y el Muestreador de Gibbs, el algoritmo EM fue descrito en la secci\'on anterior y aunque \'util para el an\'alisis del modelo presentado, el Muestreador de Gibbs tiene una interpretaci\'on m\'as simple.\\
\\
De acuerdo con \cite{gelman2014bayesian}, la simulaci\'on a trav\'es de las cadenas de Markov tambi\'en llamadas Cadenas de Markov v\'ia simulaci\'on Monte Carlo (MCMC, por sus siglas en ingl\'es) es un m\'etodo basado en las realizaciones de los valores de $\theta$ de distribuciones aproximadas y despu\'es corrigiendo esas realizaciones para poder mejorar la distribuci\'on posterior de $p(\theta|y)$. Estas realizaciones se realizan de manera iterativa. Una de estas simulaciones MCMC es el Muestreador de Gibbs.\\
\\
Normalmente el Muestreador Gibbs se asocia con la estad\'istica bayesiana aunque puede ser \'util tambi\'en en la visi\'on cl\'asica de la estad\'istica, seg\'un \cite{casella1992explaining} este algoritmo es una t\'ecnica que genera variables aleatorias indirectamente de distribuciones marginales sin tener que calcular la densidad, debido a que se basa en las propiedades principales de las Cadenas de Markov como la estacionareidad para simplificar c\'alculos y tener estimados m\'as precisos.\\
\\
Siguiendo la ilustraci\'on de \cite{casella1992explaining}, supongamos que tenemos una distribuci\'on conjunta $f(\theta,y_1,y_2,...,y_p)$\\
\[f(\theta)=\int \cdots \int f(\theta,y_1,y_2,...,y_p) dy_1,dy_2,...,dy_p\]
Si el inter\'es se encuentra en la marginal $f(\theta)$ y \'esta es demasiado complicada para calcularse directamente, con el Muestreador de Gibb se puede generar una muestra $\theta_1,...,\theta_m \sim f(\theta)$ sin la necesidad de calcular la distribuci\'on marginal. Esto permite obtener informaci\'on de la misma con alto grado de precisi\'on.\\
\\
Para ejemplificar mejor el mecanismo del Muestreador de Gibbs se toman dos variables aleatorias $(\Theta,Y)$. El algoritmo genera una muestra de $f(\theta)$ muestreando de las distribuciones condicionales $f(\theta|y)$ y $f(y|\theta)$ que son la que normalmente se conocen en los modelos estad\'isticos. Esta muestra se obtiene mediante, lo que \cite{casella1992explaining} nombra como, una secuencia de Gibbs $(Y'_0,\theta'_0,Y'_1,\theta'_1,...,Y'_k,\theta'_k)$ que de manera iterativa genera variables aleatorias a partir de valores iniciales especificados $(Y'_0=y'_0)$.\\
\\El proceso iterativo es como sigue\\
\begin{align*}
\theta'_j \sim f(\theta|Y'_j=y'_j)\\
Y'_{j+1} \sim f(y|\theta'_j=x'_j)
\end{align*}
Si la muestra es suficientemente grande, es decir, que si $k \rightarrow \infty$ la distribuci\'on de $\theta'_k$ converger\'a con la verdadera distribuci\'on marginal de $\theta$.\\
\\
El Muestreador de Gibbs puede pensarse como una implementaci\'on pr\'actica del concepto de que solo conociendo las distribuciones marginales se puede determinar la distribici\'on conjunta. Esto ser\'ia cierto en la mayor\'ia de los casos bivariados, el procedamiento no es tan directo para los casos multivariados.\\
\\
De acuerdo con \cite{casella1992explaining} para el caso bivariado, suponemos dos variables aleatorias $\theta,Y$, de las cuales se conocen sus distribuciones condicionales $f_{\Theta|Y}(\theta|y)$ y $f_{Y|\Theta}(y|\theta)$. A partir de estas podr\'iamos calcular la funci\'on marginal de $\theta$ y la distribuci\'on conjunta de ambas variables, mediante  el siguiente argumento:\\
\[f_\theta(\theta)=\int f_{\theta Y}(\theta,y)dy\]
donde la distribuci\'on conjunta es a\'un desconocida, tomando el hecho que $f_{\theta Y}(\theta,y)=f_{\theta|Y}(\theta|y)f_Y(y)$ tendr\'iamos que,\\
\[f_\theta(\theta)=\int f_{\theta|Y}(\theta|y)f_Y(y) dy\]
Asimismo, si sustituimos la distribuci\'on marginal de $y$ ($f_Y(y)$) con el mismo argumento utilizado para la distribuci\'on marginal de $\theta$, se tiene que
\begin{eqnarray*}
f_\theta(\theta) &=& \int f_{\theta|Y}(\theta|y) f_{Y|\theta}(y|t) f_\theta(t)dt dy\\
       &=& \int [ \int  f_{\theta|Y}(\theta|y)f_{Y|\theta}(y|t) dy]  f_\theta(t) dt
\end{eqnarray*}
Esta ecuaci\'on es una forma limitada de la iteraci\'on de Gibbs, ilustrando como las distribuciones condicionales producen una distribuci\'on marginal. Aunque la distribuci\'on conjunta de las variables determinan las distribuciones condicionales y marginales, no siempre las condicionales determinen de manera tan directa la distribuci\'on marginal. Esto es cierto no solo para los casos bivariados, sino que se extiende a los multivariados.\\
\\
En cuantas m\'as variables existan, el problema se vuelve m\'as complejo pues la relaci\'on entre las condicionales, marginales y conjuntas se vuelve m\'as intrincada. Por ejemplo, la relaci\'on $condicional \times marginal = conjunta$ no se sontiene para todas las condicionales y marginales. Pero se pueden hacer varios conjuntos de variables y construir las ecuaciones integrales para calcular la distribuci\'on marginal de inter\'es.\\
\\
Para casos multivariados \cite{casella1992explaining} supone las variables aleatorias $X,Y,Z$ con inter\'es en la distribuci\'on $f_X(x)$. Para esto, se toman las variables $(Y,Z)$ como una sola variable, lo que resultar\'ia en\\
\[f_X(x)= \int [ \int \int f_{X|YX}(x|y,z)f_{YZ|X}(y,z|t)dy dz] f_X(t) dt\]
De esta manera, muestreando iterativamente de $f_{X|YZ}$ y $f_{YZ|X}$ resultar\'ian en una serie de variables aleatorias que convergen en $f_X(x)$. Por otro lado, el Muestreador de Gibb muestrear\'ia iterativamente las distribuciones $f_{X|YZ}, f_{Y|XZ}, f_{Z|X}$, de tal modo que en la j-\'esima iteraci\'on se tendr\'ia,\\
\begin{align*}
X'_j \sim f(x|Y'_j = y'_j, Z'_j=z'_j)\\
Y'_{j+1} \sim f(y|X'_j=x'_j, Z'_j=z'_j)\\
Z'_{j+1} \sim f(z|X'_j=x'_j, Y'_{j+1}=y'_{j+1})
\end{align*}
Este esquema de iteraciones nos produce una secuencia de Gibbs,\\
\[Y'_0,Z'_0,X'_0,Y'_1,Z'_1,X'_1,...\]
con la misma propiedad de convergencia que en el caso bivariado, ente m\'as grande es la $k$, $X'_k=x'_k$ es un punto de la distribuci\'on marginal $f(x)$.\\
\\
De este modo queda evidenciada la utilidad del Muestreador de Gibbs en el ahorro de c\'alculos y la precisi\'on de sus resultados. Como mencionado en la secci\'on anterior, esta t\'ecnica inferencial es muy \'util tanto en la estad\'istica bayesiana como en la cl\'asica, en la primera para calcular la distribuci\'on posterior y en la \'ultima, para calcular la funci\'on de verosimilitud. Seg\'un \cite{gelman2014bayesian}, la clave del \'exito de este m\'etodo es la iteraci\'on en la cual las distribuciones aproximadas mejoran hasta converger en la distribuci\'on deseada.\\
\\
Para la implementaci\'on del Muestreador de Gibbs en este trabajo se utilizar\'a el programa JAGS (Solo Otro Muestreador de Gibbs, por sus siglas en ingl\'es). Este programa es una extensi\'on del programa BUGS (Inferencia Bayesiana Usando el Muestreador de Gibbs, por sus siglas en ingl\'es), que normalmente es utilizado para la exploraci\'on de modelos Markov multivariados en el contexto de estad\'istica actuarial. JAGS es un programa que autom\'aticamente construye las cadenas de Matkov v\'ia simulaci\'on Monte Carlo (MCMC) para modelos multivariados.\\
\\
Para lograr expresar estos problemas con BUGS, explicado por sus creadores \cite{plummer2003jags}, se necesitan tomar en cuenta dos distribuciones:
\begin{itemize}
\item Una distribuci\'on que describa la probabilidad de estar en el estado $j$ en el tiempo $t$ dado que el sujeto estaba en el estado $i$ al tiempo $0$.
\item Una distribuci\'on de supervivencia que describa el tiempo en un estado absorbente, que bien, es conocida o censorada a la derecha.
\end{itemize}
La necesidad de estas distribuciones aunado a la necesidad de una herramienta que ayude a explorar modelos gr\'aficos es lo que da origen a JAGS. Es decir, el programa toma la descripci\'on del modelo general de probabilidad multivariado y regresa un muestreo de MCMC de la distribuci\'on posterior.\\
\\
De este modo, se establece no solo las bases para la inferencia y predicci\'on de futuras observaciones en base a la verosimilud extendida y la resoluci\'on de su funci\'on sino tambi\'en la implementaci\'on num\'erica de la misma. Una vez que se definieron estas herramientas, lo que resta es la adaptaci\'on del modelo a la ilustraci\'on con los datos, esto se explorar\'a en el siguiente cap\'itulo.  