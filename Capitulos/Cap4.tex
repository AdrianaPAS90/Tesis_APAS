\chapter{Inferencia Bayesiana}
\section{Introducci\'on}
En los cap\'itulos anteriores se especificaron el modelo general de probabilidad y las distribuciones de las variables sobre las que se busca hacer inferencia. Existen varios m\'etodos de estimaci\'on que se podr\'ian utilizar en la inferencia de este trabajo de investigaci\'on, en este cap\'itulo se pretende desarrollar la justificaci\'on para la utilizaci\'on de un enfoque bayesiano para la estimaci\'on.\\
\\
En la primera secci\'on se da una breve introducci\'on al paradigma bayesiano en general, para despu\'es desarrollar sobre el muestreador de Gibbs como m\'etodo de estimaci\'on para los par\'ametros y variables latentes. De este modo se puede completar la parte te\'orica de este trabajo de investigaci\'on.  
\section{Paradigma Bayesiano}
El paradigma bayesiano se refiere a una manera de hacer inferencia basado en el trabajo del ingl\'es Thomas Bayes. En este paradigma se establece que la hip\'otesis se va actualizando de acuerdo a la nueva informaci\'on  relevante. Seg\'un \cite{gelman2014bayesian}, una de las principales razones para el pensamiento bayesiano es que facilita la interpretaci\'on basada en el sentido com\'un de conclusiones estad\'isticas. \\
%\cite{kruschke2014doing} al principio la distribuci\'on de probabilidades entre las opciones posibles se realiza al conocimiento previo que se tenga de la situaci\'on; por lo que la inferencia bayesiana consiste en la relocaci\'on o actualizaci\'on de las probabilidades entre las opciones conforme se vayan realizando nuevas observaciones.\\
\\
De acuerdo con \cite{gelman2014bayesian} la inferencia bayesiana se hace en base en una evaluaci\'on retrospectiva del procedimiento utilizado para estimar el par\'ametro sobre la distribuci\'on de todas las posibles observaciones. Es decir, que mediante la regla de Bayes se describe la relaci\'on entre la asiganci\'on previa de la probabilidad y la reasignaci\'on de esta misma condicionada a los datos observados. El paradigma est\'a basado en la regla o teorema de Bayes.\\
\\
Una definici\'on de la probabilidad condicional de $y$ dado $x$  ser\'a la divisi\'on de la funci\'on conjunta de probabilidad entre la funci\'on de probabilidad de $x$. Es decir,
\begin{align*}
p(y|x)=\frac{p(x,y)}{p(x)}
\end{align*}
O en otras palabras, la probabilidad de $y$ $suceda$ dado $x$ es la probabilidad de que $sucedan$ ambos eventos relativo a que $x$ $suceda$ en absoluto.\\
\\
Tomando en cuenta esta definici\'on de la probabilidad condicional, el teorema de Bayes, se define como
\begin{align*}
p(y|x)=\frac{p(x|y)p(y)}{p(x)}
\end{align*}
El Teorema de Bayes resulta muy \'util cuando el modelo de probabilidad se basa en variables observables y  variables latentes, $D$ y $\theta$ respectivamente. De este modo, el modelo se escribe como
\begin{align*}
p(\theta|D)=\frac{p(D|\theta)p(\theta)}{p(D)}
\end{align*}
Donde los elementos de esta ecuaci\'on significan
\begin{itemize}
\item $p(\theta|D)=$ la distribuci\'on posterior, es decir, la probabilidad de $\theta$ tomando en cuenta las observaciones.
\item $p(D|\theta)=$ la verosimilitud, es decir, la probabilidad de los datos generados por el modelo con el par\'ametro $\theta$.
\item $p(\theta)=$ la distribuci\'on previa, es decir, la probabilidad de $\theta$ sin tomar en cuenta las observaciones $D$.
\item $p(D)=$ la distribuci\'on de las observaciones, es decir, la probabilidad total de las observaciones, ponderadas por todos los valores que puede tomar el par\'ametro de acuerdo al peso que se le asigna.
\end{itemize}
Esta misma ecuaci\'on se puede reescribir como,
\begin{eqnarray*}
p(\theta|D)&=&\frac{p(D|\theta)p(\theta)}{p(D)}\\
		   &\propto & p(D|\theta)p(\theta)\\
		   &\propto & verosimilitud \times inicial.
\end{eqnarray*}
La distribuci\'on posterior est\'a en funci\'on del par\'ametro, por lo que la distribuci\'on posterior es proporcional a la multiplicaci\'on de la funci\'on de verosimilitud por la distribuci\'on inicial del par\'ametro. En otras palabras, la distribuci\'on previa del par\'ametro es la informaci\'on a priori del mismo, que se va actualizando con las observaciones, tomando la informaci\'on relevante al par\'ametro. Este mismo principio puede extenderse para varios par\'ametros.\\
\\
Como describe \cite{gelman2014bayesian}, esta l\'ogica es similar para hacer inferencias sobre futuras observaciones. Una vez que se tienen todas las observaciones $D=(d_1,...,d_n)$ se quiere inferir la siguiente observaci\'on $d_{n+1}$. La distribuci\'on de esta observaci\'on se llama la distribuci\'on posterior predictiva, posterior porque toma la informaci\'on de las observaciones pasadas y predictiva porque predice la siguiente observaci\'on,
\begin{eqnarray*}
p(d_{n+1}|D)&=&\int p(d_{n+1},\theta|D) d\theta \\
			&=& \int p(d_{n+1}|\theta,D)p(\theta|D)d\theta \\
			&=& \int p(d_{n+1}|\theta)p(\theta|D)d\theta.
\end{eqnarray*}
En la segunda y tercera l\'inea de la ecuaci\'on se muestra la distribuci\'on posterior predictiva como un promedio de las distribuciones predictivas condicionales de la distribuci\'on posterior del par\'ametro $\theta$. En la \'ultima l\'inea se asume la independencia condicional de $D$ y $d_{n+1}$ dado $\theta$.\\
\\
Tomando estas mismas ideas en un conexto m\'as general, como el planteado por \cite{goldstein2013observables}, donde la incertidumbre espist\'emica se expresa a trav\'es la distribui\'on previa del par\'ametro y la incertidumbre aleatoria como la funci\'on de verosimilitud dada por las observaciones. Por lo que se podr\'ia decir que la distribuci\'on posterior, es una mezcla entre ambas incertidumbres. En el an\'alisis estad\'istico, se necesitan construir modelos basados en los valores desconocidos de la distribuci\'on de las observaciones, es decir, de incertidumbre espist\'emica pero tomando valores de un modelo param\'etrico o de incertidumbre aleatoria; esto se logra mediante el concepto de intercambiabilidad.\\
\\
Como fue definido en el cap\'itulo dos, a trav\'es del Teorema de De Finetti; la intercambiabilidad es una propiedad de la cual se deriva que sin importar la reordenaci\'on de las observaciones, \'estas tienen la misma probabilidad de ocurrir. Las implicaciones del supuesto de intercambiabilidad, de acuerdo con \cite{goldstein2013observables}, son sorprendentemente simples. Debido a la simetr\'ia con la que se aplica el concepto de intercambiabilidad sobre las observaciones, es como si se estuviera muestreando independientemente del modelo de los valores desconocidos de las observaciones con una distribuci\'on previa sobre el par\'ametro, retomando los conceptos de incertidumbre epist\'emica y aleatoria. De este modo, mediante la observaci\'on de las variables observables y la aplicaci\'on del paradigma bayesiano, se reduce la incertidumbre al actualizar la informaci\'on sobre las variables y facilitando la inferencia.\\
\\
Otra de las mayores implicaciones del teorema de intercambiabilidad es muy intuitivo, pues la distribuci\'on de las observaciones no es mas que el resultado de todas las posibles observaciones futuras; y la divisi\'on entre los componentes epist\'emicos y aleatorios de incertidumbre de esta estructura es nuestro propio juicio sobre dichas observaciones futuras. De igual modo, esto da una entrada natural a la inferencia.\\
\\
%Como enunciado en \cite{smith2010bayesian}, en el modelo bayesiano de inferencia, como descrito anteriormente; la evidencia, las observaciones y los argumentos cient\'ificos se utilizan para soportar la distribuci\'on de probabilidad del modelo propuesto. Por otro lado, en el an\'alisis bayesiano de decisi\'on el Tomador de Decisiones debe tomar este sustento evidencial y cient\'ifico para resolver el problema espec\'ifico que se encuentre.\\
%\\
%Es decir, en un ambiente de incertidumbre tenemos el espacio de decisi\'on $A$ donde cualquier decisi\'on puede ser tomada por el Tomador de Decisi\'on y $\Phi$ es el espacio de posibles resultados $\phi$. El Tomador de Decisiones debe cuantificar las consecuencias de elegir cada decisi\'on $a \in A$ para todos los posibles resultados $\phi \in \Phi$. Con esta informaci\'on se deben especificar la funci\'on de p\'erdida $L(a,\phi)$, en la cual se mide la p\'erdida de tomar la decisi\'on $a$ con el resultado $\phi$. Otra funci\'on a especificar es la funci\'on de probabilidad $p(\phi)$ que da las probabilidades de los posibles resultados $\phi$ antes de tomar la decisi\'on $a$; esta funci\'on representa la incertidumbre que enfrenta el Tomador de Decisi\'on. De este modo, la mejor decisi\'on que se puede tomar es aquella que minimice la p\'erdida esperada,
%\begin{align*}
%\bar{L}(a)= \int_{\phi \in \Phi} L(a,\phi)p(\phi)
%\end{align*}
Con el enfoque del an\'alisis bayesiano los problemas cl\'asicos de inferencia como la estimaci\'on puntual, estimaci\'on por regiones y contraste de hip\'otesis pueden resolverse de esta manera. Adem\'as, los estimadores obtenidos no solo suelen coincidir con los estimadores cl\'asicos en algunos casos, sino en otros casos de hecho los mejoran.\\
\\
En el desarrollo de este trabajo de investigaci\'on, veremos que la estimaci\'on del modelo general de probabilidad no se puede realizar a trav\'es de m\'etodos anal\'iticos cerrados, por lo que se sugiere la utilizaci\'on de m\'etodos num\'ericos. El m\'etodo n\'umerico a utilizar, por la naturaleza del estudio, ser\'a el Muestreador de Gibbs, que se describe con m\'as detalle en el ap\'endice.\\
\\
Como mencionado en el cap\'itulo anterior, la estimaci\'on de par\'ametros del modelo general de probabilidad que se utilizar\'a en este trabajo no se puede hacer a trav\'es de m\'etodos anal\'iticos tradicionales, por lo que se utilizar\'an m\'etodos num\'ericos. Existen varios m\'etodos, entre estos se encuentran el algoritmo EM y el Muestreador de Gibbs (\cite{casella1992explaining}), el algoritmo EM ser\'a m\'as desarrolado en la secci\'on anterior y aunque \'util para el an\'alisis del modelo presentado, el Muestreador de Gibbs tiene una interpretaci\'on m\'as simple.\\
\\
De acuerdo con \cite{gelman2014bayesian}, la simulaci\'on a trav\'es de las cadenas de Markov tambi\'en llamadas Cadenas de Markov v\'ia simulaci\'on Monte Carlo (MCMC, por sus siglas en ingl\'es) consiste en construir una Cadena de Markov cuya distribuci\'on estacionaria (l\'imite) sea una distribuci\'on de la cual se pretenda simular. Una de estas simulaciones MCMC es el Muestreador de Gibbs.\\
%Para la implementaci\'on del Muestreador de Gibbs en este trabajo se utilizar\'a el programa JAGS (Solo Otro Muestreador de Gibbs, por sus siglas en ingl\'es). Este programa es una extensi\'on del programa BUGS (Inferencia Bayesiana Usando el Muestreador de Gibbs, por sus siglas en ingl\'es), que normalmente es utilizado para la exploraci\'on de modelos Markov multivariados en el contexto de estad\'istica actuarial. JAGS es un programa que autom\'aticamente construye las cadenas de Matkov v\'ia simulaci\'on Monte Carlo (MCMC) para modelos multivariados.\\
%\\
%Para lograr expresar estos problemas con BUGS, explicado por sus creadores \cite{plummer2003jags}, se necesitan tomar en cuenta dos distribuciones:
%\begin{itemize}
%\item Una distribuci\'on que describa la probabilidad de estar en el estado $j$ en el tiempo $t$ dado que el sujeto estaba en el estado $i$ al tiempo $0$.
%\item Una distribuci\'on de supervivencia que describa el tiempo en un estado absorbente, que bien, es conocida o censorada a la derecha.
%\end{itemize}
%La necesidad de estas distribuciones aunado a la necesidad de una herramienta que ayude a explorar modelos gr\'aficos es lo que da origen a JAGS. Es decir, el programa toma la descripci\'on del modelo general de probabilidad multivariado y regresa un muestreo de MCMC de la distribuci\'on posterior.\\
\\
De este modo, se establece no solo las bases para la inferencia y predicci\'on de futuras observaciones en base a la verosimilud extendida y la resoluci\'on de su funci\'on sino tambi\'en la implementaci\'on num\'erica de la misma. Una vez que se definieron estas herramientas, lo que resta es la adaptaci\'on del modelo a la ilustraci\'on con los datos, esto se explorar\'a m\'as adelante.  
\section{Algoritmo para el modelo de duraci\'on marcada}
Una vez que se han enunciado las nociones del paradigma bayesiano de manera general, se necesitan desarrollar las distribuciones particulares al modelo de duraci\'on marcada. En el cap\'itulo anterior se definieron las distribuciones de las variables latentes condicionadas a las observaciones. As\'i, la distribuci\'on posterior de la variable latente $\theta$ condicionada a las duraciones observadas es,
\[f_{\theta|d}(\theta|d) \propto \theta^{\alpha_d+\alpha_\theta-1} e^{\{-\theta(d+\beta_\theta)\}}\]
De igual modo, la distribuci\'on de la variable latente $\gamma$ condicionada a las duraciones y costos observados es,
\[f_{\gamma|d,c}(\gamma|d,c) \propto \left(\frac{1}{\gamma}\right)^{d+\alpha_\gamma+1} \quad e^{-\{(\frac{\beta_\gamma}{\gamma})+(\frac{c}{\gamma})^d\}}\]
De este modo, las distribuciones que resultan de los c\'alculos anteriores son la llave que se necesita para empezar a hacer inferencia, tomando estas distribuciones se redefine la verosimilitud de un solo individuo $i$ como,
\begin{eqnarray*}
&&V(\{\alpha_d,\alpha_\theta,\beta_\theta,\alpha_\gamma,\beta_\gamma\},\{\theta_i,\gamma_i\}_{j=1}^{N(T_i)}|\{d_j,c_j\}_{i=1}^I)=\\ 
&&\prod_{j=1}^{N(T_i)} Gamma(d_j|\alpha_d,\theta_j) Gamma(\theta_j|\alpha_d+\alpha_\theta,d_{j-1}+\beta_\theta)\times \\
&&Weibull(c_j|d_j,\gamma_j)\left(\frac{1}{\gamma}\right)^{d_{ij-1}+\alpha_\gamma+1}e^\{-(\frac{\beta_\gamma}{\gamma}+(\frac{c_{j-1}}{\gamma})^{d_{ij}})\} \times \\
&&\times \pi(\alpha_d)\pi(\alpha_\theta)\pi(\beta_\theta)\pi(\alpha_\gamma)\pi(\beta_\gamma)
\end{eqnarray*}
\\
Donde para las variables latentes asociada a las duraciones,
\begin{eqnarray*}
\pi(\theta_{ij}|\alpha_d,\alpha_\theta,\beta_\theta)&\propto& Gamma(d_{ij}|\alpha_d,\theta_{ij})\times\\ &\times& Gamma(\theta_{ij}|\alpha_d+\alpha_\theta,d_{ij-1}+\beta_\theta)\\
\\
&=&\frac{\theta_{ij}^{\alpha_d}}{\Gamma(\alpha_d)} d_{ij}^{\alpha_d-1} e^{\{-\theta_{ij} d_{ij}\}}\times\\      &\times& \frac{(d_{ij-1}+\beta_\theta)^{\alpha_d+\alpha_\theta}}{\Gamma(\alpha_d+\alpha_\theta)}\theta_{ij}^{\alpha_d+\alpha_\theta}e^{\{d_{ij-1}+\beta_\theta\}}\\
\\
&=&\frac{d_{ij}^{\alpha_d-1}(d_{ij-1}+\beta_\theta)^{\alpha_d+\alpha_\theta}}{\Gamma(\alpha_d)\Gamma(\alpha_d+\alpha_\theta)} \theta_{ij}^{2\alpha_d+\alpha_\theta-1}\times \\
&\times& e^{\{-\theta_{ij}(d_{ij-1}+d_{ij}+\beta_\theta)\}}\\
\\
&\propto& \theta_{ij}^{2\alpha_d+\alpha_\theta-1}e^{\{-\theta_{ij}(d_{ij-1}+d_{ij}+\beta_\theta)\}}\\
\\
&\Rightarrow& \theta_{ij} \sim Gamma(2\alpha_d+\alpha_\theta,d_{ij}+d_{ij-1}+\beta_\theta)
\end{eqnarray*}
%\begin{multline*}
%\pi(\theta_{ij}|\alpha_d,\alpha_\theta,\beta_\theta)\propto Gamma(d_{ij}|\alpha_d,\theta_{ij})\times\\ \times Gamma(\theta_{ij}|\alpha_d+\alpha_\theta,d_{ij-1}+\beta_\theta)\\
%\\
%=\frac{\theta_{ij}^{\alpha_d}}{\Gamma(\alpha_d)} d_{ij}^{\alpha_d-1} e^{\{-\theta_{ij} d_{ij}\}}\times \frac{(d_{ij-1}+\beta_\theta)^{\alpha_d+\alpha_\theta}}{\Gamma(\alpha_d+\alpha_\theta)}\theta_{ij}^{\alpha_d+\alpha_\theta}e^{\{d_{ij-1}+\beta_\theta\}}\\
%\\
%=\frac{d_{ij}^{\alpha_d-1}(d_{ij-1}+\beta_\theta)^{\alpha_d+\alpha_\theta}}{\Gamma(\alpha_d)\Gamma(\alpha_d+\alpha_\theta)} \theta_{ij}^{2\alpha_d+\alpha_\theta-1}\times e^{\{-\theta_{ij}(d_{ij-1}+d_{ij}+\beta_\theta)\}}\\
%\\
%\propto \theta_{ij}^{2\alpha_d+\alpha_\theta-1}e^{\{-\theta_{ij}(d_{ij-1}+d_{ij}+\beta_\theta)\}}\\
%\\
%\Rightarrow \theta_{ij} \sim Gamma(2\alpha_d+\alpha_\theta,d_{ij}+d_{ij-1}+\beta_\theta)
%\end{multline*}
Y para la variable latente asociada a los costos,
\begin{eqnarray*}
\pi(\gamma_{ij}|\alpha_\gamma,\beta_\gamma,d_{ij},c_{ij-1})&\propto& Weibull(c_{ij}|d_{ij},\gamma_{ij})\times\\
&\times& \bigg(\frac{1}{\gamma_{ij}}\bigg)^{d_{ij-1} +\alpha_\gamma+1} \quad e^{-\big((\frac{\beta_\gamma}{\gamma_{ij}})+(\frac{c_{ij-1}}{\gamma_{ij}})^{d_{ij-1}}\big)}\\
\\
&=&\frac{d_{ij}}{\gamma_{ij}^{d_{ij}}}\quad c_{ij}^{d_{ij-1}}\quad e^{-\big(\frac{c_{ij}}{\gamma_{ij}}\big)^{d_{ij}}}\times\\
&\times& \bigg(\frac{1}{\gamma_{ij}}\bigg)^{d_{ij-1} +\alpha_\gamma+1} \quad e^{-\big((\frac{\beta_\gamma}{\gamma_{ij}})+(\frac{c_{ij-1}}{\gamma_{ij}})^{d_{ij-1}}\big)}\\
\\
&=&d_{ij}\quad c_{ij}^{d_{ij}-1}\quad\bigg(\frac{1}{\gamma_{ij}}\bigg)^{d_{ij}+d_{ij-1}+\alpha_\gamma+1}\\
\\
&\times& e^{-\big((\frac{\beta_\gamma}{\gamma_{ij}})+(\frac{c_{ij-1}}{\gamma_{ij}})^{d_{ij-1}}+(\frac{c_{ij}}{\gamma_{ij}})^{d_{ij}}\big)}\\
\\
&\propto&\left(\frac{1}{\gamma_{ij}}\right)^{d_j+d_{ij-1}+\alpha_\gamma+1}e^{-\big((\frac{\beta_\gamma}{\gamma_{ij}})+(\frac{c_{ij-1}}{\gamma_{ij}})^{d_{ij-1}}+(\frac{c_{ij}}{\gamma_{ij}})^{d_{ij}}\big)}
\end{eqnarray*}
De manera an\'aloga, podemos desarrollar las distribuciones posteriores de los par\'ametros mediante la noci\'on bayesiana a que \'esta es proporcional la verosimilitud de las variables latentes correspondientes, multiplic\'andolos por la distribuci\'on inicial de cada par\'ametro. En primer lugar tenemos las distribuciones relativas a los par\'ametros del modelo Gamma-Gamma correspondiente a la variable observable de las duraciones $\alpha_d, \alpha_\theta, \beta_\theta$. El par\'ametro $\alpha_d$ es el par\'ametro de forma de la distribuci\'on de la variable observable de las duraciones ,\\
\begin{eqnarray*}
\pi(\alpha_d|(\theta_{ij})_{j=1}^{n_i}._{i=1}^I) &\propto& \Big(\prod_{i=1}^I Gamma(\theta_{i1}|\alpha_d+\alpha_\theta, d_{i1}+\beta_\theta) \\
&\times& \prod_{i=1}^I \prod_{j=2}^{n_i} Gamma(\theta_{ij}|2\alpha_d+\alpha_\theta, d_{ij}+d_{ij-1}+\beta_\theta) \Big) \\
&\times& Gamma(\alpha_d|\alpha_0,\beta_0)\\
\end{eqnarray*}
\\
\begin{eqnarray*}
&=& \bigg(\prod_{i=1}^I \frac{(d_{i1}+\beta_\theta)^{\alpha_d+\alpha_\theta}}{\Gamma(\alpha_d+\alpha_\theta)} \quad \theta_{i1}^{\alpha_d+\alpha_\theta-1}\quad e^{-\theta_{i1}(d_{i1}+\beta_\theta)} \\
&\times& \prod_{i=1}^I \prod_{j=2}^{n_i} \frac{(d_{ij}+d_{ij-1}+\beta_\theta)^{2\alpha_d+\alpha_\theta}}{\Gamma(2\alpha_d+\alpha_\theta)} \quad \theta_{ij}^{2\alpha_d+\alpha_\theta-1} \quad e^{-\theta_{ij}(d_{ij}+d_{ij-1}+\beta_\theta)}\bigg)\\
\\
&\times& \frac{\beta_0^{\alpha_0}}{\Gamma(\alpha_0)} \alpha_d^{\alpha_0-1} e^{-\alpha_d\beta_0}\\
\\
\\
&\propto& \Bigg(\prod_{i=1}^I \frac{(d_{i1}+\beta_\theta)^{\alpha_d}}{\Gamma(\alpha_d+\alpha_\theta)} \quad \theta_{i1}^{\alpha_d} \quad \prod_{i=1}^I \prod_{j=2}^{n_i} \frac{(d_{ij}+d_{ij-1}+\beta_\theta)^{2\alpha_d}}{\Gamma(2\alpha_d+\alpha_\theta)}\quad\theta_{ij}^{2\alpha_d}\Bigg) \times \alpha_d^{\alpha_0-1} \quad e^{-\alpha_d\beta_0}\\
\\
\\
&\propto& \Big(\frac{1}{\Gamma(\alpha_d+\alpha_\theta)}\Big)^I \Big(\frac{1}{\Gamma(2\alpha_d+\alpha_\theta)}\Big)^{\sum_{i=1}^I n_i} \quad \alpha_d^{\alpha_0-1} e^{-\alpha_d\beta_0} \times \\
&\times& \Bigg(\prod_{i=1}^I (d_{i1}+\beta_\theta)^{\alpha_d} \quad \theta_{i1}^{\alpha_d}\quad \times \prod_{i=1}^I \prod_{j=2}^{n_i} (d_{ij}+d_{ij-1}+\beta_\theta)^{2\alpha_d}\quad \theta_{ij}^{2\alpha_d}\Bigg)
\end{eqnarray*}
\\
\\
El par\'ametro $\alpha_\theta$ es el par\'ametro de forma de la distribuci\'on de la variable latente $\theta$, es por esto que la distribuci\'on posterior es tan similar a aquella de $\alpha_d$ al ser los dos par\'ametros de forma,
\begin{eqnarray*}
\pi(\alpha_\theta|(\theta_{ij})_{j=1}^{n_i}._{i=1}^I) &\propto& \Bigg(\prod_{i=1}^I Gamma(\theta_{i1}|\alpha_d+\alpha_\theta, d_{i1}+\beta_\theta)\\
&\times& \prod_{i=1}^I \prod_{j=2}^{n_i} Gamma(\theta_{ij}|2\alpha_d+\alpha_\theta, d_{ij}+d_{ij-1}+\beta_\theta) \Bigg) \\
&\times& Gamma(\alpha_\theta|\alpha_0,\beta_0)\\
\end{eqnarray*}

\begin{eqnarray*}
&=&\Bigg(\prod_{i=1}^I \frac{(d_{i1}+\beta_\theta)^{\alpha_d+\alpha_\theta}}{\Gamma(\alpha_d+\alpha_\theta)} \quad \theta_{i1}^{\alpha_d+\alpha_\theta-1} \quad e^{-\theta_{i1}(d_{i1}+\beta_\theta)}\\
&\times& \prod_{i=1}^I \prod_{j=2}^{n_i} \frac{(d_{ij}+d_{ij-1}+\beta_\theta)^{2\alpha_d+\alpha_\theta}}{\Gamma(2\alpha_d+\alpha_\theta)} \quad \theta_{ij}^{2\alpha_d+\alpha_\theta-1} \quad e^{-\theta_{ij}(d_{ij}+d_{ij-1}+\beta_\theta)}\Bigg) \\
\\
&\times & \frac{\beta_0^{\alpha_0}}{\Gamma(\alpha_0)} \quad \alpha_\theta^{\alpha_0-1} e^{-\alpha_\theta\beta_0}\\
\\
\\
&\propto& \Bigg(\prod_{i=1}^I \frac{(d_{i1}+\beta_\theta)^{\alpha_\theta}}{\Gamma(\alpha_d+\alpha_\theta)} \quad \theta_{i1}^{\alpha_\theta} \quad \prod_{i=1}^I \prod_{j=2}^{n_i} \frac{(d_{ij}+d_{ij-1}+\beta_\theta)^{\alpha_\theta}}{\Gamma(2\alpha_d+\alpha_\theta)}\quad\theta_{ij}^{\alpha_\theta}\Bigg) \times \alpha_\theta^{\alpha_0-1} \quad e^{-\alpha_\theta\beta_0}\\
\\
\\
\\
&=&\Big(\frac{1}{\Gamma(\alpha_d+\alpha_\theta)}\Big)^I \Big(\frac{1}{\Gamma(2\alpha_d+\alpha_\theta)}\Big)^{\sum_{i=1}^I n_i} \quad \alpha_\theta^{\alpha_0-1} e^{-\alpha_\theta\beta_0} \times \\
&\times& \Bigg(\prod_{i=1}^I (d_{i1}+\beta_\theta)^{\alpha_\theta} \quad \theta_{i1}^{\alpha_\theta}\quad \prod_{i=1}^I \prod_{j=2}^{n_i} (d_{ij}+d_{ij-1}+\beta_\theta)^{\alpha_\theta} \quad \theta_{ij}^{\alpha_\theta}\Bigg)
\end{eqnarray*}
\\
\\
El par\'ametro $\beta_\theta$ es el par\'ametro de escala correspondiente a la distribuci\'on de la variable latente de las duraciones, $\theta$,
\begin{eqnarray*}
\pi(\beta_\theta|(\theta_{ij})_{j=1}^{n_i}._{i=1}^I) &\propto& \Big(\prod_{i=1}^I Gamma(\theta_{i1}|\alpha_d+\alpha_\theta, d_{i1}+\beta_\theta)\\
&\times& \prod_{i=1}^I \prod_{j=2}^{n_i} Gamma(\theta_{ij}|2\alpha_d+\alpha_\theta, d_{ij}+d_{ij-1}+\beta_\theta) \Big) \\
&\times& Gamma(\beta_\theta|\alpha_0,\beta_0)\\
\end{eqnarray*}
\\
\begin{eqnarray*}
&=&\Bigg(\prod_{i=1}^I \frac{(d_{i1}+\beta_\theta)^{\alpha_d+\alpha_\theta}}{\Gamma(\alpha_d+\alpha_\theta)} \quad \theta_{i1}^{\alpha_d+\alpha_\theta-1} \quad e^{-\theta_{i1}(d_{i1}+\beta_\theta)} \\
&\times& \prod_{i=1}^I \prod_{j=2}^{n_i} \frac{(d_{ij}+d_{ij-1}+\beta_\theta)^{2\alpha_d+\alpha_\theta}}{\Gamma(2\alpha_d+\alpha_\theta)} \quad \theta_{ij}^{2\alpha_d+\alpha_\theta-1} \quad e^{-\theta_{ij}(d_{ij}+d_{ij-1}+\beta_\theta)}\Bigg) \\
\\
&\times & \frac{\beta_0^{\alpha_0}}{\Gamma(\alpha_0)} \quad \beta_\theta^{\alpha_0-1} e^{-\beta_\theta\beta_0}\\
\\
\\
&\propto& \Bigg(\prod_{i=1}^I (d_{i1}+\beta_\theta)^{\alpha_d+\alpha_\theta} \quad e^{-\theta_{i1}\beta_\theta} \quad \times \prod_{i=1}^I \prod_{j=2}^{n_i} (d_{ij}+d_{ij-1}+\beta_\theta)^{2\alpha_d+\alpha_\theta} \quad e^{-\theta_{ij}\beta_\theta} \Bigg) \\
\\
&\times& \beta_\theta^{\alpha_0-1} \quad e^{\beta_\theta \beta_0}\\
\\
&=& \Bigg(\prod_{i=1}^I (d_{i1}+\beta_\theta)^{\alpha_d+\alpha_\theta}\times \prod_{i=1}^I \prod_{j=2}^{n_i} (d_{ij}+d_{ij-1}+\beta_\theta)^{2\alpha_d+\alpha_\theta}\quad \Bigg)\\
&\times& \beta_\theta^{\alpha_0-1} \quad e^{-\beta_\theta\big(\beta_0+\sum_{i=1}^I \sum_{j=1}^{n_i}\theta_{ij}\big)}
\end{eqnarray*}
\\
\\
\\
De igual modo, tenemos las distribuciones para los par\'ametros del modelo Gamma Inversa-Weibull que describe el comportamiento de los costos a trav\'es de la variable latente $\gamma$: $\alpha_\gamma, \beta_\gamma$. El par\'ametro $\alpha_\gamma$ es el par\'ametro de forma correspondiente a la distribuci\'on Inversa Gamma que describe la variable latente,\\
\begin{eqnarray*}
\pi(\alpha_\gamma|(\gamma_{ij})_{i=1}^I._{j=1}^{n_i}) &\propto& \Bigg(\prod_{i=1}^I \bigg(\frac{1}{\gamma_{i1}}\bigg)^{d_{i1+\alpha_\gamma+1}}\quad e^{-\big((\frac{\beta_\gamma}{\gamma_{i1}})+(\frac{c_{i1}}{\gamma_{i1}})^{d_{i1}}\big)}\\
&\times& \prod_{i=1}^I \prod_{j=2}^{n_i} \bigg(\frac{1}{\gamma_{ij}}\bigg)^{d_{ij}+d_{ij-1}+\alpha_\gamma+1}\quad e^{-\big((\frac{\beta_\gamma}{\gamma_{ij}})+(\frac{c_{ij-1}}{\gamma_{ij}})^{d_{ij-1}}+(\frac{c_{}ij}{\gamma_{ij}})^{d_{ij}}\big)}\Bigg)\\
&\times& Gamma(\alpha_\gamma|\alpha_0,\beta_0)
\end{eqnarray*}
\\
\begin{eqnarray*}
&=&\Bigg(\prod_{i=1}^I \bigg(\frac{1}{\gamma_{i1}}\bigg)^{d_{i1+\alpha_\gamma+1}}\quad e^{-\big((\frac{\beta_\gamma}{\gamma_{i1}})+(\frac{c_{i1}}{\gamma_{i1}})^{d_{i1}}\big)}\\
&\times& \prod_{i=1}^I \prod_{j=2}^{n_i} \bigg(\frac{1}{\gamma_{ij}}\bigg)^{d_{ij}+d_{ij-1}+\alpha_\gamma+1}\quad e^{-\big((\frac{\beta_\gamma}{\gamma_{ij}})+(\frac{c_{ij-1}}{\gamma_{ij}})^{d_{ij-1}}+(\frac{c_{ij}}{\gamma_{ij}})^{d_{ij}}\big)}\Bigg)\\
\\
&\times& \frac{\beta_0^{\alpha_0}}{\Gamma(\alpha_0)} \quad \alpha_\gamma^{\alpha_0-1} e^{-\alpha_\gamma\beta_0}\\
\\
\\
&\propto& \Bigg(\prod_{i=1}^I \Big(\frac{1}{\gamma_{i1}}\Big)^{\alpha_\gamma} \times \prod_{i=1}^I \prod_{j=2}^{n_i}\Big(\frac{1}{\gamma_{ij}}\Big)^{\alpha_\gamma}\Bigg)\quad \alpha_\gamma^{\alpha_0-1}e^{-\alpha_\gamma\beta_0}\\
\\
\\
&=&\Bigg(\prod_{i=1}^I \prod_{j=1}^{n_i} \Big(\frac{1}{\gamma_{ij}}\Big)^{\alpha_\gamma}\Bigg) \quad \alpha_\gamma^{\alpha_0-1}e^{-\alpha_\gamma\beta_0}
\end{eqnarray*}
\\
\\
Mientras que el par\'ametro $\beta_\gamma$ corresponde al par\'ametro de escala correspondiente a la distribuci\'on Inversa Gamma que describe el comportamiento de la variable latente, $\gamma$,\\
\begin{eqnarray*}
\pi(\beta_\gamma|(\gamma_{ij})_{i=1}^I._{j=1}^{n_i}) &\propto& \Bigg(\prod_{i=1}^I \bigg(\frac{1}{\gamma_{i1}}\bigg)^{d_{i1+\alpha_\gamma+1}}\quad e^{-\big((\frac{\beta_\gamma}{\gamma_{i1}})+(\frac{c_{i1}}{\gamma_{i1}})^{d_{i1}}\big)}\\
&\times& \prod_{i=1}^I \prod_{j=2}^{n_i} \bigg(\frac{1}{\gamma_{ij}}\bigg)^{d_{ij}+d_{ij-1}+\alpha_\gamma+1}\quad e^{-\big((\frac{\beta_\gamma}{\gamma_{ij}})+(\frac{c_{ij-1}}{\gamma_{ij}})^{d_{ij-1}}+(\frac{c_{ij}}{\gamma_{ij}})^{d_{ij}}\big)}\Bigg) \\
\\
&\times& Gamma(\beta_\gamma|\alpha_0,\beta_0)
\end{eqnarray*}
\\
\begin{eqnarray*}
&=&\Bigg(\prod_{i=1}^I \bigg(\frac{1}{\gamma_{i1}}\bigg)^{d_{i1+\alpha_\gamma+1}}\quad e^{-\big((\frac{\beta_\gamma}{\gamma_{i1}})+(\frac{c_{i1}}{\gamma_{i1}})^{d_{i1}}\big)}\\
&\times& \prod_{i=1}^I \prod_{j=2}^{n_i} \bigg(\frac{1}{\gamma_{ij}}\bigg)^{d_{ij}+d_{ij-1}+\alpha_\gamma+1}\quad e^{-\big((\frac{\beta_\gamma}{\gamma_{ij}})+(\frac{c_{ij-1}}{\gamma_{ij}})^{d_{ij-1}}+(\frac{c_{ij}}{\gamma_{ij}})^{d_{ij}}\big)}\Bigg)\\
\\
&\times& \frac{\beta_0^{\alpha_0}}{\Gamma(\alpha_0)} \quad \beta_\gamma^{\alpha_0-1} e^{-\beta_\gamma\beta_0}\\
\\
\\
&\propto& \Bigg(\prod_{i=1}^I e^{-\big(\frac{\beta_\gamma}{\gamma_{i1}}\big)}\prod_{i=1}^I \prod_{j=2}^{n_i} e^{-\big(\frac{\beta_\gamma}{\gamma_{ij}}\big)}\Bigg) \quad \beta_\gamma^{\alpha_0-1} e^{-\beta_\gamma\beta_0}\\
\\
&=&e^{-\bigg(\sum_{i=1}^I \sum_{j=1}^{n_i}\big(\frac{\beta_\gamma}{\gamma_{ij}}\big)\bigg)}\quad \times \beta_\gamma^{\alpha_0-1}\quad e^{-\beta_\gamma\beta_0}\\
\\
&=& \beta_\gamma^{\alpha_0-1} e^{-\beta_\gamma\big(\beta_0+\sum_{i=1}^I \sum_{j=1}^{n_i}\big(\frac{1}{\gamma_{ij}}\big))}\\
\\
&\propto& \beta_\gamma \sim Gamma\Bigg(\alpha_0,\beta_0+\sum_{i=1}^I \sum_{j=1}^{n_i}\big(\frac{\beta_\gamma}{\gamma_{ij}}\big)\Bigg)
\end{eqnarray*}
\\
\\
Es importante mencionar que las distribuciones resultantes de la verosimilitud extendida cumplen con las propiedades de log-concavidad descritas en el cap\'itulo anterior. De este modo y de  manera general para cada individuo $i$, tomando la verosimilitud extendida definida anteriormente se fijan los valores iniciales para los par\'ametros $\alpha_d^{(0)},\alpha_\theta^{(0)},\beta_\theta^{(0)},\alpha_\gamma^{(0)},\beta_\gamma^{(0)}$ y para las variables latentes $\{\theta_j^{(0)}\}_{j=1}^{n_i},\{\gamma_j^{(0)}\}_{j=1}^{n_i}$ y para cada $k=1,...,n_i$ tenemos la siguiente distribuci\'on que es proporcional a la verosimilitud,
\begin{align*}
\pi(\alpha_d^{(k)},\alpha_\theta^{(k)},\beta_\theta^{(k)},\alpha_\gamma^{(k)},\beta_\gamma^{(k)}|\{\theta_i^{(k-1)}\}_{i=2}^{n_i},\{\gamma_i^{(k-1)}\}_{i=2}^{n_i},\{d_i,c_i\}_{i=1}^{n_i})
\end{align*}
Este es el principio necesario para utilizar el Muestreador de Gibbs de modo que se estimen los par\'ametros en base a las variables latentes que a su vez se estiman en base a las observaciones para que con los par\'ametros estimados se estimen las variables latentes que ayuden a predecir futuras observaciones. %Una vez que se tiene el concepto general de la aplicaci\'on de esta t\'ecnica de muestreo para el objetivo de este trabajo, es necesario considerar las maneras de implementarlo.\\

Una vez que se definen las distribuciones particulares del modelo y los m\'etodos num\'ericos con los que se realizar\'a la inferencia, en el siguiente cap\'itulo se implementar\'a con una base de datos espec\'ifica. Es importante mencionar que, independientemente de la base de datos con la que se trabaje, las distribuciones y la forma del algoritmo es la correspondiente al modelo general de probabilidad.